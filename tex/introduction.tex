\section{Introduction}

In today's world, Neural Networks have experienced tremendous growth, finding applications in diverse areas such as facial recognition and language translation, thus enhancing our daily lives. Although significant progress has been made in many research areas, there remains untapped potential for improvement in machine learning on graphs. Within this context, Machine Learning (ML) for chemical sciences is gaining increased attention and has become the focus of a rising number of research efforts. As a result, the collection and labeling of molecular data have surged, which in turn leads to a wealth of information available for training ML models. These models are poised to further advance chemical discovery and molecular analysis, ultimately benefiting areas such as drug development and molecular investigation.

\subsection{Problem Domain}

To facilitate the enhancement of ML models, benchmarks have been established. In a machine learning context, a benchmark serves as a platform for comparing models against a collection of datasets, which are divided into fixed training and testing sets. The primary benchmark for this project is MoleculeNet, which has sought to standardize dataset splitting by introducing multiple ways to divide the datasets that can be applied universally. Since the MoleculeNet paper was published in 2018, our objective is to identify and employ methods that have been utilized in other contexts but not in the paper.

\subsection{Definitions}

A graph is a mathematical structure that consists of a set of vertices $V$ and a set of edges $E$.
Similarly a attributed graph is a mathematical structure that consists of a set of vertices and a set of edges, where each vertex and edge has an attribute, e.g. $A(V \bigcup E) \to \Sigma^*$.
While this basic definition does not specify any conditions on the attributes, it is possible to limit the attributes to a certain set of values.
Some graph machine learning methods require the attributes to be of certain types or to provide certain functionality.
